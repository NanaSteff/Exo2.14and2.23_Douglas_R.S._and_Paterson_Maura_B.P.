\documentclass[12pt,a4paper]{article}
\usepackage[latin1]{inputenc}			% Encoding
\usepackage{amsmath}
\usepackage{amsfonts}
\usepackage{amssymb}
\usepackage{graphicx}
\usepackage{multirow,multicol}
\usepackage{pifont,hyperref,lastpage,fancyhdr,movie15,float}
\usepackage{eqnarray}
\usepackage{caption}
\usepackage{subcaption}
\usepackage[total={170mm,257mm}, left=20mm, top=20mm]{geometry}

%%%%%%%%%%%%%%%%%%%%%%%%%%% Footer and Header %%%%%%%%%%%%%%%%%%%%%%%%%%%%%%%%%%%%%%%
\pagestyle{fancy}                                                     				%
\fancyhf{} 		                                                     				%
\lfoot{\tiny\textsf{\includegraphics[width=1.9cm]{Figures/AIMSCameroonlogo}         %
Po.Box~608 Limbe,~phone~{(+237) 233333363},~                          				%
\url{http://www.aims-cameroon.org}}}                                  				%
\rfoot{\bfseries Page \thepage~of \pageref{LastPage}}                 				%
\renewcommand{\footrulewidth}{2.pt}\renewcommand{\headrulewidth}{0pt} 				%
%%%%%%%%%%%%%%%%%%%%%%%%%%%%%%%%%%%%%%%%%%%%%%%%%%%%%%%%%%%%%%%%%%%%%%%%%%%%%%%%%%%%%

%%%%%%%% Fill here information about the present assignment %%%%%%%%%%%%%%%%%%%%%%%
\newcommand{\code}{\textbf{iMTR815ja}}                                        %
\newcommand{\deadline}{05.03.23, 11:59 pm}                                         %
\newcommand{\assignment}{assignment 1 ON Cryptography}				           	   % 
\newcommand{\lecturer}{Lecturer(s): Prof. Giulio Codogni}                   			   %
%%%%%%%%%%%%%%%%%%%%%%%%%%%%%%%%%%%%%%%%%%%%%%%%%%%%%%%%%%%%%%%%%%%%%%%%%%%%%%%%%%%%

%%%%%%%%%%%%% Title at the first page %%%%%%%%%%%%%%%%%%%%%%%%%%%%%%%%%%%%%%%%%%%%%%%%%%%
\title{\vspace*{-2cm}\begin{minipage}{\textwidth}                                       %
\begin{center}                                                                          %
\begin{tabular}{|c|c|c|}                                                                %
\hline\multicolumn{3}{|c|}{\bf\scriptsize\MakeUppercase\assignment}\\                   %
\hline{\small Student's Code}&                                                          %
\multirow{3}{7cm}{\includegraphics[width=7.1cm,height=1.4cm]{Figures/AIMSCameroonlogo}} %
& {\small Deadline}\\                                                                   %
\cline{1-1}\cline{3-3}{\small\bf\code}&&{\small\bf\deadline} \\       					%
\cline{1-1}\cline{3-3}{\small\today} &&{\small Ac. Year: 2022 - 2023}\\           					%
\hline\multicolumn{3}{|r|}{\scriptsize\lecturer}\\\hline              					%
\end{tabular}                                                         					%
\end{center}                                                          					%
\end{minipage}\hfill\date{}\vspace*{-1cm}}                            					%
%%%%%%%%%%%%%%%%%%%%%%%%%%%%%%%%%%%%%%%%%%%%%%%%%%%%%%%%%%%%%%%%%%%%%%%%%%%%%%%%%%%%%%%%%

\newcommand{\K}{\mathbb{K}}
\newcommand{\R}{\mathbb{R}}
\newcommand{\C}{\mathbb{C}}
\newcommand{\D}{\mathbb{D}}
\newcommand{\Z}{\mathbb{Z}}
\newcommand{\N}{\mathbb{N}}

\newtheorem{theo}{Theorem}
\newtheorem{defi}{Definition}
\newenvironment{proof}[1][Proof.]{\textbf{#1~}}{\ \rule{0.5em}{0.5em}}

\begin{document}
\maketitle\thispagestyle{fancy}

\section{Solution of Exercise 2.14}

\textbf{a)-}Let's prove that $\det A\equiv\pm 1 (\mod 26)$ if $A$ is a matrix over $\Z/26$ such that $A=A^{-1}$.

Let's suppose that $A$ is a matrix over $\Z/26$ such that $A=A^{-1}$

As $A=A^{-1}$ so $A.A=A^{-1}.A=Id$. So we have $A^2=Id$.

As we have $\det A^2=(\det A)^2=\det Id=1$ so $\det A=\pm 1$ then we have $\det A\equiv\pm 1 (\mod 26)$.

\textbf{b)-}Let's use the formula given in Corollary 2.4 to determine the number of involutory keys in the Hill Cipher (over $\Z/26$) in the case $m=2$.


According to Corollary 2.4, 
Suppose
\[A=\begin{pmatrix}
a&b\\c&d
\end{pmatrix}\]
is a matrix having entries in $\Z_n$ , and $\det K=ad-cb$ is invertible in $\Z_n$. Then
\[A^{-1}=(\det A)^{-1}\begin{pmatrix}
	d&-b\\-c&a
\end{pmatrix}\]

To find the number of involutary keys in the Hill Cipher, we are looking for all the matrices $A$ such that $A=A^{-1}$ and it means that all the matrices A whose $\det A=\pm 1$.

According to the corollary, 
\[A^{-1}=A=(\det A)^{-1}\begin{pmatrix}
	d&-b\\-c&a
\end{pmatrix}\qquad\text{where }\det A=\pm 1\]

\[\begin{pmatrix}
	a&b\\c&d
\end{pmatrix}=\pm \begin{pmatrix}
	d&-b\\-c&a
\end{pmatrix}\]


\section*{Case $\det A=1$}

\[\begin{pmatrix}
	a&b\\c&d
\end{pmatrix}=\begin{pmatrix}
	d&-b\\-c&a
\end{pmatrix}\]

So we have $a=d$, $b=-b$, $c=-c$. 
\begin{align}
b=-b&\implies 2b=0 \mod 26\implies b=13 \text{ or } b=0\\
c=-c&\implies 2c=0 \mod 26\implies c=13 \text{ or } c=0
\end{align}
Let's solve \begin{align}
	\det A=ad-cb=a^2-cb=1 \label{eq 1}
\end{align}

If $c=b=0$ so we have $a^2=1\implies a=\pm 1$ 2 possibilities, which are 
\[\begin{pmatrix}
	1&0\\0&1
\end{pmatrix}\qquad\begin{pmatrix}
	25&0\\0&25
\end{pmatrix}\]
If $c=13$, $b=0$ so we have $a^2=1\implies a=\pm 1$ with have 2 possibilities 
\[\begin{pmatrix}
	1&0\\13&1
\end{pmatrix}\qquad\begin{pmatrix}
	25&0\\13&25
\end{pmatrix}\]
and same for $c=0$ and $b=13$ 
\[\begin{pmatrix}
	1&13\\0&1
\end{pmatrix}\qquad\begin{pmatrix}
	25&13\\0&25
\end{pmatrix}\]
If $c=13$, $b=13$ implies \begin{align}
a^2&-13^2=1\\
a^2&=1+13\\
a^2&=14\implies a=14 \text{ or } a=12 \text{ which is 2 possibilities}
\end{align} 
\[\begin{pmatrix}
	14&13\\13&14
\end{pmatrix}\qquad\begin{pmatrix}
	12&13\\13&12
\end{pmatrix}\]
So we have 8 matices over $\Z_{26}$ such that $\det A=1$ for each matrix $A$.


\section*{Case $\det A=-1$}

\[\begin{pmatrix}
	a&b\\c&d
\end{pmatrix}=-\begin{pmatrix}
	d&-b\\-c&a
\end{pmatrix}=\begin{pmatrix}
-d&b\\c&-a
\end{pmatrix}\]
So we have $a=-d$
Let's solve \begin{align}
	\det A=ad-cb=a^2-cb=-1 \label{eq 2}
\end{align}

As 26 is a product of 2 and 13 which are two prime numbers so 
\[\Z_{26}\to\Z_{2}\times\Z_{13}\]

In $\Z_{2}$, $a=-d\implies a=d=0$ or $a=d=1$.\\\\ If $a=0$ then $ad-cb=-1=1\implies c=d=1$, which is $\begin{pmatrix}0&1\\1&0 \end{pmatrix}$

 If $a=1$ then $cb=0\implies c=0$ or $b=0$ or $c=b=0$. So we have 4 possibilities in $\Z_{2}$, which are
 \[\begin{pmatrix}
 	1&0\\0&1
 \end{pmatrix}\qquad\begin{pmatrix}
 	1&0\\0&1
 \end{pmatrix}\qquad\begin{pmatrix}
 	1&0\\1&1
 \end{pmatrix}\]

In $\Z_{13}$, $ad-cb=-1$ with $a=-d$ implies $a^2+bc=1$.\\

If $a^2=1$ so we have $bc=0$
First case, if $a=1$ and $b=0$ so we have 12 choices of $c$ such that $c\neq 0$. And same if $a=1$ and $c=0$, we have 12 choices of $b$ such that $b\neq 0$. The last case is $a=0$, $b=0$ and also $c=0$. So we have 25 possibilities for $a=1$.

If $a=-1$, we have the same possibility as in $a=1$. 
So for $a=\pm 1$, we have $2\times 25$ possibilities.\\\\     
If $a\neq\pm 1$ so there is 11 possibilities of $a$. And as $\Z_{13}$ is a field so $cb\in\Z_{13}$ and has 12 possibilities in $\Z_{13}^*$. So for $a\neq\pm 1$ we have $11\times 12$ possibility of matrices. 

So, there are $25\times 2+11\times 12=182$ matrices of determinants equal to $-1$ in $\Z_{13}$ so in $\Z_{2}\times\Z_{13}$ there are $4\times 182$ matrices of determinant equal to $-1$ which is the same number in $\Z_{26}$.

To conclude, the number of involutory key in the Hill Cipher is the sum of the number of matrix of determinant equal to 1 and the number of matrix of determinant equal to $-1$ which is equal to $8+728=736$. So there are 736 involutory keys.

\section{Solution of Exercise 2.23}
Suppose we are told that the plaintext
\begin{center}
\texttt{breathtaking}
\end{center}
yields the ciphertext
\begin{center}
\texttt{RUPOTENTOIFV}
\end{center}
where the Hill Cipher is used (but m is not specified). Let's determine the encryption matrix.
\begin{center}
$\begin{array}{c c c c c c c c c c c c c}
	a&b&c&d&e&f&g&h&i&j&k&l&m\\
	0&1&2&3&4&5&6&7&8&9&10&11&12\\\\
	n&o&p&q&r&s&t&u&v&w&x&y&z\\
	13&14&15&16&17&18&19&20&21&22&23&24&25
\end{array}$
\end{center}
\section*{The plaintext}
\begin{center}
$\begin{array}{c c c c c c c c c c c c}
b&r&e&a&t&h&t&a&k&i&n&g\\
1&17&4&0&19&7&19&0&10&8&5&21
\end{array}$
\end{center}
\section*{The ciphertext}
\begin{center}
$\begin{array}{c c c c c c c c c c c c}
R&U&P&O&T&E&N&T&O&I&F&V\\
17&20&15&14&19&4&13&19&14&8&5&21
\end{array}$
\end{center}
As the length of our message is 12 so the possible value of m is a divisor of 12 so m may be equal to 1,2,3,4 or 6.

\section*{If m=1}
So, $e_K(1)=17\implies 17=1.K\implies K=17$\\
We also have that $e_K(17)=20$ so $17K=20\implies 17^2=20\mod 26$ which is false so $m\neq 1$

\section*{If m=2}
So, 
\begin{align}
e_K(1~17)&=(17~20)\\
e_K(4~0)&=(15~14)
\end{align}
From the first two plaintext-ciphertext pair, we get the matrix equation 
\[\begin{pmatrix}
1&17\\4&0
\end{pmatrix}=\begin{pmatrix}
17&20\\15&14
\end{pmatrix}.K\]
So $K=\begin{pmatrix}
	17&20\\15&14
\end{pmatrix}^{-1}.\begin{pmatrix}
1&17\\4&0
\end{pmatrix}$ or $\det\begin{vmatrix}
17&20\\15&14
\end{vmatrix}=17\times 14-15\times 20=10$ which is not coprime to 26 so the matrix is not invertible. So $m\neq 2$.

\section*{If m=3}
So, 
\begin{align}
	e_K(1~17~4)&=(17~20~15)\\
	e_K(0~19~7)&=(14~19~4)\\
	e_K(19~0~10)&=(13~19~14)
\end{align}
From the first three plaintext-ciphertext, we get the matrix equation 
\[\begin{pmatrix}
	17&20&15\\14&19&4\\13&19&14
\end{pmatrix}=\begin{pmatrix}
	1&17&4\\0&19&7\\19&0&10
\end{pmatrix}.K\]

So $K=\begin{pmatrix}
	1&17&4\\0&19&7\\19&0&10
\end{pmatrix}^{-1}.\begin{pmatrix}
17&20&15\\14&19&4\\13&19&14
\end{pmatrix}$ where $\begin{pmatrix}
1&17&4\\0&19&7\\19&0&10
\end{pmatrix}^{-1}=\begin{pmatrix}
10&2&5\\7&2&1\\7&17&1
\end{pmatrix}$\\

So 
\begin{align}
K&=\begin{pmatrix}
	10&2&5\\7&2&1\\7&17&1
\end{pmatrix}.\begin{pmatrix}
17&20&15\\14&19&4\\13&19&14
\end{pmatrix}\\
K&=\begin{pmatrix}
3&21&20\\4&15&23\\6&14&5
\end{pmatrix}
\end{align}

\end{document}

